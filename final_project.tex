% Options for packages loaded elsewhere
\PassOptionsToPackage{unicode}{hyperref}
\PassOptionsToPackage{hyphens}{url}
%
\documentclass[
]{article}
\usepackage{amsmath,amssymb}
\usepackage{lmodern}
\usepackage{ifxetex,ifluatex}
\ifnum 0\ifxetex 1\fi\ifluatex 1\fi=0 % if pdftex
  \usepackage[T1]{fontenc}
  \usepackage[utf8]{inputenc}
  \usepackage{textcomp} % provide euro and other symbols
\else % if luatex or xetex
  \usepackage{unicode-math}
  \defaultfontfeatures{Scale=MatchLowercase}
  \defaultfontfeatures[\rmfamily]{Ligatures=TeX,Scale=1}
\fi
% Use upquote if available, for straight quotes in verbatim environments
\IfFileExists{upquote.sty}{\usepackage{upquote}}{}
\IfFileExists{microtype.sty}{% use microtype if available
  \usepackage[]{microtype}
  \UseMicrotypeSet[protrusion]{basicmath} % disable protrusion for tt fonts
}{}
\makeatletter
\@ifundefined{KOMAClassName}{% if non-KOMA class
  \IfFileExists{parskip.sty}{%
    \usepackage{parskip}
  }{% else
    \setlength{\parindent}{0pt}
    \setlength{\parskip}{6pt plus 2pt minus 1pt}}
}{% if KOMA class
  \KOMAoptions{parskip=half}}
\makeatother
\usepackage{xcolor}
\IfFileExists{xurl.sty}{\usepackage{xurl}}{} % add URL line breaks if available
\IfFileExists{bookmark.sty}{\usepackage{bookmark}}{\usepackage{hyperref}}
\hypersetup{
  pdftitle={Trends in Avocado Prices},
  pdfauthor={Nicole Tresvalles \& Karen Galvan},
  hidelinks,
  pdfcreator={LaTeX via pandoc}}
\urlstyle{same} % disable monospaced font for URLs
\usepackage[margin=1in]{geometry}
\usepackage{graphicx}
\makeatletter
\def\maxwidth{\ifdim\Gin@nat@width>\linewidth\linewidth\else\Gin@nat@width\fi}
\def\maxheight{\ifdim\Gin@nat@height>\textheight\textheight\else\Gin@nat@height\fi}
\makeatother
% Scale images if necessary, so that they will not overflow the page
% margins by default, and it is still possible to overwrite the defaults
% using explicit options in \includegraphics[width, height, ...]{}
\setkeys{Gin}{width=\maxwidth,height=\maxheight,keepaspectratio}
% Set default figure placement to htbp
\makeatletter
\def\fps@figure{htbp}
\makeatother
\setlength{\emergencystretch}{3em} % prevent overfull lines
\providecommand{\tightlist}{%
  \setlength{\itemsep}{0pt}\setlength{\parskip}{0pt}}
\setcounter{secnumdepth}{-\maxdimen} % remove section numbering
\ifluatex
  \usepackage{selnolig}  % disable illegal ligatures
\fi

\title{Trends in Avocado Prices}
\author{Nicole Tresvalles \& Karen Galvan}
\date{Last updated on 2022-04-05}

\begin{document}
\maketitle

{
\setcounter{tocdepth}{2}
\tableofcontents
}
\includegraphics[width=1\linewidth]{https://images.squarespace-cdn.com/content/v1/5661b263e4b0830bdc162128/1531234150802-7M3VMHK28W1T2DF5POZN/5901c94f2600001700c47fbc}

\begin{center}\rule{0.5\linewidth}{0.5pt}\end{center}

\hypertarget{introduction}{%
\section{Introduction}\label{introduction}}

For this final project, we are interested in the the trends in Hass
Avocado sales in different regions of the US and how sales change over
time. According to the USDA, the United States has tripled their per
capita consumption of avocados since the early 2000's, with about 95\%
of avocados sold being Hass avocados. Along with its taste, size, and
longer shelf life, Hass avocados have had greater growing yield in
certain regions\footnote{Global hass avocado market: 2022 - 27: Industry
  share, size, growth. Global Hass Avocado Market \textbar{} 2022 - 27
  \textbar{} Industry Share, Size, Growth - Mordor Intelligence. (n.d.).
  Retrieved March 9, 2022, from
  \href{https://www.mordorintelligence.com/industry-reports/global-hass-avocado-market\#:~:text=In\%20the\%20United\%20States\%2C\%20per,many\%20countries\%20across\%20the\%20globe}{here}.}.
The information under the following dataset is directly reported by
retailers describing the price of avocados and the amount they sell.
We've filtered the dataset as to only include reports from the
Northeast, West, and Plains regions of the US. In addition to region of
avocado sales, the data we work with also provide information about
date, year, total volume of avocado sales, as well as the type of
avocado involved. The dataset reports that type differentiates organic
and conventional avocados, which is determined the amount of detectable
levels of pesticides residue.

\begin{center}\rule{0.5\linewidth}{0.5pt}\end{center}

\hypertarget{exploratory-data-analysis}{%
\section{Exploratory data analysis}\label{exploratory-data-analysis}}

\hypertarget{describe-data}{%
\subsection{Describe data}\label{describe-data}}

Give people a sense for your data by describing:

\begin{enumerate}
\def\labelenumi{\arabic{enumi}.}
\tightlist
\item
  What each observation in each row corresponds to: Each observation
  corresponds to the average price of a single avocado being sold in a
  given date and region.
\item
  Sample size: There's a total of 18,259 observations included in the
  dataset
\item
  Outcome variable \(y\): The average price of an avocado
\item
  Numerical explanatory variable \(x_1\): Total volume of avocados sold
\item
  Categorical explanatory variable \(x_2\) and what the \(k\)
  levels/categories are: Date and year of observation, Region of the
  observation (which has 6 categories), type of avocado
  (organic/conventional)
\item
  Display a snapshot of five randomly chosen rows of data
\end{enumerate}

\begin{verbatim}
## # A tibble: 5 x 6
##   Date       AveragePrice Total.Volume type          year region   
##   <date>            <dbl>        <dbl> <chr>        <dbl> <chr>    
## 1 2018-02-11         1.11      617476. conventional  2018 Northeast
## 2 2015-11-08         1.12     1285587. conventional  2015 Plains   
## 3 2018-03-18         1.34     4621126. conventional  2018 Northeast
## 4 2016-08-21         1.82      144812. organic       2016 West     
## 5 2017-03-19         1.43      380613. conventional  2017 Northeast
\end{verbatim}

\hypertarget{summary-statistics}{%
\subsection{Summary statistics}\label{summary-statistics}}

\begin{verbatim}
## # A tibble: 2 x 7
##   term      estimate std_error statistic p_value lower_ci upper_ci
##   <chr>        <dbl>     <dbl>     <dbl>   <dbl>    <dbl>    <dbl>
## 1 intercept  -64.8       20.7      -3.13   0.002 -105.     -24.2  
## 2 year         0.033      0.01      3.20   0.001    0.013    0.053
\end{verbatim}

\begin{verbatim}
## # A tibble: 3 x 7
##   term           estimate std_error statistic p_value lower_ci upper_ci
##   <chr>             <dbl>     <dbl>     <dbl>   <dbl>    <dbl>    <dbl>
## 1 intercept         1.54      0.013    118.         0    1.51     1.56 
## 2 region: Plains   -0.103     0.023     -4.57       0   -0.147   -0.059
## 3 region: West     -0.267     0.023    -11.8        0   -0.312   -0.223
\end{verbatim}

\hypertarget{data-visualizations}{%
\subsection{Data visualizations}\label{data-visualizations}}

\begin{figure}

{\centering \includegraphics{final_project_files/figure-latex/unnamed-chunk-8-1} 

}

\caption{Figure 1: Types of Avocado Average Prices in 2016}\label{fig:unnamed-chunk-8}
\end{figure}

The median average price for organic avocados in 2016 is much greater
than the median average price for conventional avocados. The
approximated difference seems to be \$0.50.

\begin{figure}

{\centering \includegraphics{final_project_files/figure-latex/unnamed-chunk-9-1} 

}

\caption{Figure 2: Avocado Average Prices in 2016}\label{fig:unnamed-chunk-9}
\end{figure}

Consistently across all regions in 2016, the average price of an organic
avocado is greater than its conventional counter part. Specifically, the
Northeast and Plains regions seem to have stronger correlation
coefficients than their counterpart.

\begin{figure}

{\centering \includegraphics{final_project_files/figure-latex/unnamed-chunk-10-1} 

}

\caption{Figure 3: Total Volume of Avocado Sold in 2016}\label{fig:unnamed-chunk-10}
\end{figure}

The median average price of avocados sold in 2016 seems to be much lower
in the West than in other regions of the US, with the Northeast having
the highest median average price.

\begin{figure}

{\centering \includegraphics{final_project_files/figure-latex/unnamed-chunk-11-1} 

}

\caption{Figure 4: Average Price of Total Volume Sold in 2016 (Interaction)}\label{fig:unnamed-chunk-11}
\end{figure}

\begin{verbatim}
## # A tibble: 6 x 7
##   term                      estimate std_error statistic p_value lower_ci upper_ci
##   <chr>                        <dbl>     <dbl>     <dbl>   <dbl>    <dbl>    <dbl>
## 1 intercept                    1.58      0.02      80.7    0        1.54     1.62 
## 2 Total.Volume                 0         0         -8.14   0        0        0    
## 3 region: Plains               0.07      0.038      1.82   0.069   -0.005    0.145
## 4 region: West                -0.107     0.039     -2.77   0.006   -0.183   -0.031
## 5 Total.Volume:regionPlains    0         0         -8.29   0        0        0    
## 6 Total.Volume:regionWest      0         0         -1.10   0.271    0        0
\end{verbatim}

This visualization gives us the insight that the avocado prices in
Northeast averages much higher per pound than avocados in other regions.
Throughout the regions, as the total volume sold increases, the average
price of the avocados decrease. A significant note is that in the
Plains, the trend decreases much quicker compared to the other regions.

\begin{figure}

{\centering \includegraphics{final_project_files/figure-latex/unnamed-chunk-12-1} 

}

\caption{Figure 5: Average Price of Total Volume Sold in 2016 (Parallel)}\label{fig:unnamed-chunk-12}
\end{figure}

\begin{verbatim}
## # A tibble: 4 x 7
##   term           estimate std_error statistic p_value lower_ci upper_ci
##   <chr>             <dbl>     <dbl>     <dbl>   <dbl>    <dbl>    <dbl>
## 1 intercept          1.60     0.019     84.4    0        1.57     1.64 
## 2 Total.Volume       0        0        -15.6    0        0        0    
## 3 region: Plains    -0.14     0.031     -4.56   0       -0.2     -0.079
## 4 region: West      -0.11     0.033     -3.32   0.001   -0.176   -0.045
\end{verbatim}

Like the previous model, this visualization gives us the insight that
the avocado prices in Northeast averages higher per pound than avocados
in other regions, followed by avocados in the West, Midsouth, South
East, the Plains, and Northern New England. Furthermore, as the total
volume of avocados sold increases, the average price decreases.

\hypertarget{initial-conclusions}{%
\subsection{Initial conclusions}\label{initial-conclusions}}

Write:

\begin{itemize}
\tightlist
\item
  What insight you would tell your grandmother
\item
  Your initial model selection choice, interaction or parallel slopes
  model, and why.
\end{itemize}

From the initial glance at the visualizations, we conclude that organic
avocados tend to consistently cost more on average across all the
regions we highlighted. Furthermore, as the total volume of avocados
sold increases, the average price gets less, which might be because of
wholesale. As for the model selection, it seems like the parallel slopes
model would be the best choice because it is easier to interpret. It has
a lot less variation compared to the interaction model that has
different intercepts and different slopes. Furthermore, the interaction
model has additional complexity which can be seen by the additional rows
in the interaction regression table.

\begin{center}\rule{0.5\linewidth}{0.5pt}\end{center}

\hypertarget{multiple-linear-regression}{%
\section{Multiple linear regression}\label{multiple-linear-regression}}

\hypertarget{regression-table}{%
\subsection{Regression table}\label{regression-table}}

Instructions:

\begin{itemize}
\tightlist
\item
  Indicate which model you selected: parallel slopes or interaction.
\item
  Fit your selected regression model and display the regression table in
  the code block below
\item
  Modify the template formula below from ModernDive 6.1.2 to match your
  model (this formula is written in the LaTeX typesetting language for
  printing mathematical formulas)
\end{itemize}

\[
\widehat{y} = 
4.883 
- 0.018 \cdot \text{age}  
- 0.446 \cdot \mathbb{1}_{\text{is male}}(x)
+ 0.014 \cdot \text{age} \cdot \mathbb{1}_{\text{is male}}(x)
\]

\hypertarget{interpreting-regression-coefficients}{%
\subsection{Interpreting regression
coefficients}\label{interpreting-regression-coefficients}}

Interpret all regressions coefficients in the \texttt{estimate} column
in the following list:

\begin{enumerate}
\def\labelenumi{\arabic{enumi}.}
\tightlist
\item
  Interpretation 1
\item
  Interpretation 2
\item
  \ldots{}
\end{enumerate}

\hypertarget{inference-for-multiple-regression}{%
\subsection{Inference for multiple
regression}\label{inference-for-multiple-regression}}

\textbf{You will complete the rest of Section 3 at the project
resubmission phase due the last day of exams.}

From the above regression table, interpret the confidence interval and
p-value for at least two non-intercept terms:

\begin{enumerate}
\def\labelenumi{\arabic{enumi}.}
\tightlist
\item
  Interpretation 1
\item
  Interpretation 2
\item
  (optional) Interpretation 3
\item
  (optional) \ldots{}
\end{enumerate}

\hypertarget{conditions-for-inference-for-regression}{%
\subsection{Conditions for inference for
regression}\label{conditions-for-inference-for-regression}}

Verify the \textbf{LINE} conditions for inference for regression for all
confidence intervals and p-values from your regression table to have
valid interpretation

\hypertarget{linearity-of-relationship}{%
\subsubsection{Linearity of
relationship}\label{linearity-of-relationship}}

\hypertarget{independence-of-residuals}{%
\subsubsection{Independence of
residuals}\label{independence-of-residuals}}

\hypertarget{normality-of-residuals}{%
\subsubsection{Normality of residuals}\label{normality-of-residuals}}

\hypertarget{equality-of-variance}{%
\subsubsection{Equality of variance}\label{equality-of-variance}}

\begin{center}\rule{0.5\linewidth}{0.5pt}\end{center}

\hypertarget{discussion}{%
\section{Discussion}\label{discussion}}

\hypertarget{conclusions}{%
\subsection{Conclusions}\label{conclusions}}

\hypertarget{limitations}{%
\subsection{Limitations}\label{limitations}}

\hypertarget{further-questions}{%
\subsection{Further questions}\label{further-questions}}

\begin{center}\rule{0.5\linewidth}{0.5pt}\end{center}

\hypertarget{honor-code}{%
\section{Honor code}\label{honor-code}}

\hypertarget{project-peers}{%
\subsection{Project peers}\label{project-peers}}

Name all people who contributed in any way to this project (other than
groupmates, Prof.~Kim, and Beth Brown): Not applicable

\hypertarget{code-sources}{%
\subsection{Code sources}\label{code-sources}}

List any sources for coding matters you consulted in bullet point form
(other than SDS 220 materials). For example, you can change the
following list:

\begin{enumerate}
\def\labelenumi{\arabic{enumi}.}
\tightlist
\item
  \href{https://www.datanovia.com/en/blog/ggplot-legend-title-position-and-labels/}{Labeling
  legends in ggplot2}
\item
  \href{https://www.datasciencemadesimple.com/select-random-samples-r-dplyr-sample_n-sample_frac/\#:~:text=Dplyr\%20package\%20in\%20R\%20is,the\%20random\%20N\%25\%20of\%20rows.}{Randomly
  sampling rows with dplyr}
\item
  \href{https://stackoverflow.com/questions/23564607/how-to-change-x-axis-tick-label-names-order-and-boxplot-colour-using-r-ggplot}{Changing
  x-label tick marks}
\end{enumerate}

\hypertarget{citations-and-references}{%
\subsection{Citations and References}\label{citations-and-references}}

Ensure all in-text citations (if any) show up here using RMarkdown
footnotes as seen in the Introduction:

\end{document}
